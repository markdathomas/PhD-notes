\documentclass{article}
\usepackage{graphicx} % Required for inserting images
\usepackage{amsmath}
\usepackage{amssymb}  
\usepackage{amsfonts} 
\usepackage{xcolor}\usepackage{geometry} % Required for customizing page layout
\usepackage{hyperref}
 \usepackage{mdframed}

% Set custom page width
\geometry{
    a4paper, % or letterpaper (US)
    left=2cm, % Adjust the left margin
    right=2cm, % Adjust the right margin
    top=2cm, % Adjust the top margin
    bottom=2cm, % Adjust the bottom margin
}


\title{PhD notes}
\author{Mark Thomas}
\date{January 2024}

\begin{document}

\maketitle


\section{Summary of variations of different terms possible in a Lagrangian}
Making sure to recall the Hodge operator rule: $\lambda \wedge \omega = \omega \wedge * \lambda$. The first result is
\begin{equation}
    \label{Eq: Maxwell term variation}
    -\frac{1}{2}\delta\left(\mathrm{d}a_{e}\wedge *\mathrm{d}a_{e} \right) = \delta a_{e}\wedge \mathrm{d} * \mathrm{d}a_{e}- \mathrm{d}\left(\delta{a}_{e}\wedge * \mathrm{d}a_{e} \right)
\end{equation}
as derived in appendix \ref{appendixsourcefreemaxwell}. Next is the Abelian Chern-Simons term, which is derived in appendix \ref{appendixabelianchernsimonsvariation}, gives the result
\begin{equation}
    \label{Eq: Chern simons term variation}
    \delta\left(\frac{k}{4\pi}a_{e}\wedge \mathrm{d}a_{e} \right) = \frac{k}{2\pi}\delta a_{e}\wedge \mathrm{d}a_{e} + \frac{k}{4\pi}\mathrm{d}\left(a_{e}\wedge \delta a_{e}\right).
\end{equation}


\section{The Abelian Maxwell Chern-Simons Lagrangian}
The Lagrangian of this theory is 
\begin{equation}
    L_{\text{MCS}} = -\frac{1}{2g^{2}} \mathrm{d}a_{e}\wedge * \mathrm{d}a_{e} + \frac{k}{4\pi}a_{e}\wedge \mathrm{d}a_{e},\quad a_{e}\in \mathfrak{u_{1}}
\end{equation}
This has the equation of motion (derived in appendix \ref{appendixamcseom}) of 
\begin{equation}
    \mathrm{d} * \mathrm{d} a_{e} + \frac{g^{2}k}{2\pi}\mathrm{d}a_{e}=0,
\end{equation}
meaning it has a mass $M = \frac{g^{2}k}{2\pi}$. This is Yang Mills where the source is from the Chern-Simons term. 

\section{Some results on the path integrals of fields}
Much like the integral
\begin{equation}
    \int e^{-\frac{1}{2}ax^{2}+bx}\mathrm{d}x = \sqrt{\frac{2\pi}{a}}e^{\frac{b^{2}}{2a}},
\end{equation}
by dropping the unnecessary factors out front, the path integral 
\begin{equation}
    \boxed{
    \int_{\mathfrak{g}\times M}\mathcal{D}F\exp i \int_{M}\left(-\frac{g^{2}}{2}F\wedge * F + F\wedge \mathrm{d}A \right)\sim \exp i \int_{M}\left(-\frac{1}{2g^{2}}\mathrm{d}A\wedge *\mathrm{d}A\right).
    }
\end{equation}
This is fleshed out fully in the appendix \ref{appendixfirstpathintegral}. Similarly, the integral
\begin{equation}
    \boxed{
    \int \mathcal{D}F\mathcal{D}A\exp i \int_{M}\left(-\frac{g^{2}}{2}F\wedge * F + F\wedge \mathrm{d}A \right)\sim \int \mathcal{D}F\delta\left(\mathrm{d}F \right)\exp i \int_{M}\left(-\frac{g^{2}}{2}F\wedge * F \right).
    }
\end{equation}

\section{Abelian master partition function}
Consider the ``master'' partition function
\begin{equation}
    \mathcal{Z} = \int \mathcal{D}a_{m}\mathcal{D}a_{e}\,\exp i \int \left( -\frac{g^{2}}{2}a_{m}\wedge *a_{m} + a_{m}\wedge \mathrm{d}a_{e} + \frac{k}{4\pi}a_{e}\wedge \mathrm{d}a_{e} \right),
\end{equation}
with $a_{m}$ a gauge invariant one-form. The Lagrangian for this is
\begin{equation}
    L =  -\frac{g^{2}}{2}a_{m}\wedge *a_{m} + a_{m}\wedge \mathrm{d}a_{e} + \frac{k}{4\pi}a_{e}\wedge \mathrm{d}a_{e}.
\end{equation}
This varies as
\begin{equation}
    \delta L = \delta a_{m}\wedge \left(g^{2}\mathrm{d}*\mathrm{d}a_{m} + \mathrm{d}a_{e}\right) + \delta a_{e}\wedge \left(\mathrm{d}a_{m} + \frac{k}{2\pi}\mathrm{d}a_{e}\right) + \mathrm{d}\left(g^{2}*\mathrm{d}a_{m}\wedge \delta a_{m} + \frac{k}{4\pi}a_{e}\wedge \delta a_{e}\right)
\end{equation}
upon the variation of both $a_{m}$ and $a_{e}$ independently. This is the result of varying the action \textit{before} any integration over the fields. 


\section{Non-Abelian master partition function for MCS and its dual}
The theory is
\begin{align}
    \label{Eq: Master equation}
    \mathcal{Z}_{\text{Master}}&= \int \mathcal{D}a_{m}\mathcal{D}a_{e}\exp i \text{tr}\left\{\int \left(-\frac{g^{2}}{2}a_{m}\wedge * a_{m} \right.\right. + a_{m}\wedge \left(\mathrm{d}a_{e} + a_{e}\wedge a_{e} \right) \nonumber \\
    &+ \left.\left.\frac{k}{4\pi}\left(a_{e}\wedge \mathrm{d}a_{e} + \frac{2}{3}a_{e}\wedge a_{e}\wedge a_{e} \right)\right)\right\}.
\end{align}
The Lagrangian for this is
\begin{equation}
    L = -\frac{g^{2}}{2}a_{m}\wedge * a_{m}  + a_{m}\wedge \left(\mathrm{d}a_{e} + a_{e}\wedge a_{e} \right)+ \frac{k}{4\pi}\left(a_{e}\wedge \mathrm{d}a_{e} + \frac{2}{3}a_{e}\wedge a_{e}\wedge a_{e} \right).
\end{equation}
This varies to give
\begin{equation}
    Put the stuff here.
\end{equation}



To get the electric theory (MCS), integrate over $a_{m}$ (see Appendix \ref{appendixmagneticintegration}) to get
\begin{equation}
    Z_{\text{Electric}} = \int \mathcal{D}a_{m}\exp i \text{tr}\left\{\int \text{stuff}\right\}.
\end{equation}
By substituting out the usual expression
\begin{equation}
    a_{e} = b - \left(\frac{2\pi}{k} \right)a_{m}, 
\end{equation}
get the (non-integrated) magnetic theory
\begin{align}
    \label{Eq: Expanded magnetic}
    Z_{\text{Magnetic}} = \int \mathcal{D}a_{m}\mathcal{D}b \exp i \text{tr}\bigg\{\int &-\frac{g^{2}}{2}a_{m}\wedge *a_{m} - \frac{\pi}{k}a_{m}\wedge \mathrm{d}a_{m} + \frac{k}{4\pi}b\wedge \mathrm{d}b \nonumber\\
    &- \left(\frac{2\pi}{k}\right)a_{m}\wedge b \wedge a_{m}\nonumber\\
    &+ \frac{2}{3}\left(\frac{2\pi}{k}\right)^{2}a_{m}\wedge a_{m}\wedge a_{m}+ \frac{k}{6\pi}b\wedge b \wedge b\bigg\},
\end{align}
which is derived in \ref{appendixelectricsubstitution}. Can organise this in various ways, for example
\begin{equation}
    \boxed{
    Z_{\text{M}} = \int \mathcal{D}a_{m}\mathcal{D}b\exp i \text{tr}\bigg\{\int -\frac{g^{2}}{2}a_{m}\wedge *a_{m} - \frac{\pi}{k}a_{m}\wedge \left[\mathrm{d}a_{m}+2a_{m}\wedge \left(b - \frac{4\pi}{3k}a_{m}\right)\right] + \frac{k}{4\pi}b\wedge \left(\mathrm{d}b + \frac{2}{3}b\wedge b\right),
    }
\end{equation}
Might be useful to compare this to the Abelian case, where
\begin{equation}
    Z_{M}^{(A)} = \int \mathcal{D}a_{m}\mathcal{D}b \exp i \int \left\{-\frac{g^{2}}{2}a_{m}\wedge *a_{m} - \frac{\pi}{k}a_{m}\wedge \mathrm{d}a_{m} + \frac{k}{4\pi}b\wedge \mathrm{d}b\right\}.
\end{equation}
Interested in seeing the relationship as to when $a_{m}$ and $b$ couple with each other. Perhaps also non-zero $b$ will change the apparent value of $k$ and so on. Can draw Feynman diagrams for the theory we get. Aim for now is to get things as tidied up as possible. Things can get really confusing as to understand what we're looking at we must understand strongly coupled gauge theories. 



\section{Checking the gauge invariance of $Z_{M}$}
Under the gauge transformations, have that $a_{m}\rightarrow g a_{m} g^{\dagger}$ as it is not a gauge field, and $b\rightarrow gbg^{\dagger} + g\mathrm{d}g^{\dagger}$. Since we're considering $g\in \mathfrak{su}(N)$, have that $g^{\dagger} g = g g^{\dagger} = \mathbf{1}$, and also $\text{tr}\left(g\right)=0$.\newline


\noindent Using the properties of $g$, the action in $Z_{M}$ is gauge invariant up to the level of the trace and total derivatives. This is easy to see for most terms in equation \eqref{Eq: Expanded magnetic}, with the exception of $a_{m}\wedge b \wedge a{m}$. Shown in appendix \ref{Sec: Behaviour of Z_{M} under gauge transformations} is that under the gauge transformation, 
\begin{equation}
    \label{Eq: AAB gauge change}
    \text{tr}\left(a_{m}\wedge a_{m}\wedge b\right)\rightarrow \text{tr}\left(a_{m}\wedge a_{m}\wedge b\right) + \text{tr}\left(\mathrm{d}\left(a_{m}\wedge a_{m}\right)\cdot g\right).
\end{equation}
We can then use the Cauchy-Schwarz inequality for the trace which states
\begin{equation}
    0\leq \left[\text{tr}\mathbf{AB}\right]^{2}\leq \text{tr}\left(\mathbf{A}^{2}\right)\text{tr}\left(\mathbf{B}^{2}\right)\leq \left[\text{tr}\left(A\right)\right]^{2}\left[\text{tr}\left(B\right)\right]^{2},
\end{equation}
which implies that if $B$ is traceless, then $\text{tr}\left(\mathbf{AB}\right)=0$. So since $g$ is traceless, the second term in \eqref{Eq: AAB gauge change} is zero, and so $\text{tr}\left(a_{m}\wedge a_{m}\wedge b\right)\rightarrow \text{tr}\left(a_{m}\wedge a_{m}\wedge b\right)$ under the gauge transformation.


\section{Later bits}





\appendix
\newpage
\section{Source free maxwell variation}
\label{appendixsourcefreemaxwell}
The variation can be calculated as
\begin{align}
    \delta\left(\mathrm{d}a_{e} \wedge * \mathrm{d}a_{e}\right) &= \mathrm{d}\left(\delta a_{e} \right)\wedge * \mathrm{d}a_{e} + \mathrm{d}a_{e}\wedge * \mathrm{d}\left(\delta a_{e} \right) \nonumber \color{blue}\quad \text{Product rule (over Lie algebra)} \color{black} \\
    &= \mathrm{d}\left(\delta a_{e} \right)\wedge * \mathrm{d}a_{e} + \mathrm{d}\left(\delta a_{e} \right)\wedge *\mathrm{d}a_{e}\nonumber\color{blue}\quad \lambda\wedge * \omega  = \omega \wedge * \lambda\color{black}\\
    &= 2\mathrm{d}\left(\delta a_{e}\right)\wedge * \mathrm{d}a_{e}\nonumber \\
    &= 2\mathrm{d}\left(\delta a_{e}\wedge * \mathrm{d}a_{e} \right) - 2 \delta a_{e}\wedge \mathrm{d}*\mathrm{d}a_{e}\nonumber\quad \color{blue} \text{Product rule (over manifold)} \color{black},
\end{align}
so
\begin{equation}
    \boxed{
    -\frac{1}{2}\delta\left(\mathrm{d}a_{e} \wedge * \mathrm{d}a_{e}\right) = \delta a_{e}\wedge \mathrm{d}*\mathrm{d}a_{e} - \mathrm{d}\left(\delta a_{e}\wedge *\mathrm{d}a_{e} \right)
    }
\end{equation}


\section{Abelian Chern-Simons variation}
\label{appendixabelianchernsimonsvariation}
Calculating
\begin{align}
    \delta\left(\frac{k}{4\pi}A\wedge \mathrm{d} A \right) &= \frac{k}{4\pi}\delta\left(A\wedge \mathrm{d}A\right)\nonumber\\
    &= \frac{k}{4\pi}\left[\delta A \wedge \mathrm{d}A + A\wedge \delta\left(\mathrm{d}A\right)\right]\color{blue}\text{ Product rule (Lie algebra)}\color{black}\nonumber\\
    &= \frac{k}{4\pi}\left[\delta A \wedge \mathrm{d}A + \mathrm{d}\left(A\wedge \delta A\right)-\mathrm{d}A\wedge \delta A\right]\color{blue}\text{ Product rule (Manifold)}\color{black}\nonumber\\
    &= \frac{k}{4\pi}\left[\mathrm{d}\left(A\wedge \delta A \right) + 2\delta A \wedge \mathrm{d}A\right] \color{blue}\text{ Wedge product antisymmetry}\color{black}.\nonumber 
\end{align}


\section{Abelian Maxwell Chern-Simons equation of motion}
\label{appendixamcseom}
Starting with the Action
\begin{equation}
    S_{\text{MCS}} = \int_{\mathcal{M}} \, -\frac{1}{2g^{2}} \mathrm{d}a_{e}\wedge * \mathrm{d}a_{e} + \frac{k}{4\pi}a_{e}\wedge \mathrm{d}a_{e},\quad a_{e}\in \mathfrak{u_{1}},
\end{equation}
variation of the fields leaves
\begin{align}
    \delta S_{\text{MCS}} &= \delta \left(\int_{\mathcal{M}} \, -\frac{1}{2g^{2}} \mathrm{d}a_{e}\wedge * \mathrm{d}a_{e} + \frac{k}{4\pi}a_{e}\wedge \mathrm{d}a_{e} \right)\nonumber \\
    &= \int_{M}-\frac{1}{2g^{2}}\delta\left(\mathrm{d}a_{e}\wedge * \mathrm{d}a_{e}\right)+\frac{k}{4\pi}\delta\left(a_{e}\wedge \mathrm{d}a_{e}\right)\nonumber
\end{align}
Then by equations \eqref{Eq: Maxwell term variation} and \eqref{Eq: Chern simons term variation}, 
\begin{align}
    \delta S_{\text{MCS}} &=\int_{M} \frac{1}{g^{2}}\delta a_{e}\wedge\mathrm{d}* \mathrm{d}a_{e} - \frac{1}{g^{2}}\mathrm{d}\left(\delta a_{e}\wedge * \mathrm{d}a_{e}\right) + \frac{k}{2\pi}\delta a_{e}\wedge \mathrm{d}a_{e} + \frac{k}{4\pi}\mathrm{d}\left(a_{e}\wedge \delta a_{e} \right)\nonumber\\
    &= \int_{M} \frac{1}{g^{2}}\delta a_{e}\wedge \mathrm{d}* \mathrm{d}a_{e} + \frac{k}{2\pi}\delta a_{e}\wedge \mathrm{d}a_{e} + \int_{\partial M}-\frac{1}{g^{2}}\delta a_{e}\wedge *\mathrm{d}a_{e} + \frac{k}{4\pi}a_{e}\wedge \delta a_{e}\nonumber\\
    &= \int_{M}\delta a_{e}\wedge \left(\frac{1}{g^{2}}\mathrm{d}* \mathrm{d}a_{e} + \frac{k}{2\pi}\mathrm{d}a_{e}\ \right)-\int_{\partial M}\delta a_{e}\wedge \left(\frac{1}{g^{2}}*\mathrm{d}a_{e} + \frac{k}{4\pi}a_{e}\right)\nonumber.
\end{align}
Imposing $\delta S_{\text{MCS}} = 0$ for all $\delta a_{e}$ with $\delta a_{e}=0$ on $\partial M$ yields the equation of motion.
\begin{equation}
    \frac{1}{g^{2}}\mathrm{d}*\mathrm{d}a_{e} + \frac{k}{2\pi}\mathrm{d}a_{e}=0.
\end{equation}

\section{First path integral}
\label{appendixfirstpathintegral}
\color{red}This bit needs some real attention. Want to show that 
\begin{equation}
    \boxed{
    \int_{\mathfrak{g}\times M}\mathcal{D}F\exp i \int_{M}\left(-\frac{g^{2}}{2}F\wedge * F + F\wedge \mathrm{d}A \right)\sim \exp i \int_{M}\left(-\frac{1}{2g^{2}}\mathrm{d}A\wedge *\mathrm{d}A\right)
    }
\end{equation}
convincingly
\color{black}
\section{Integrating the master theory to get the electric theory}
\label{appendixmagneticintegration}
Here is the integration of equation \eqref{Eq: Master equation}. 


\section{Substitution in the master theory to get the magnetic theory}
\label{appendixelectricsubstitution}
Substituting in $b$. The original partition fuction is
\begin{align}
    \mathcal{Z}_{\text{Master}}&= \int \mathcal{D}a_{m}\mathcal{D}a_{e}\exp i \text{tr}\left\{\int \left(-\frac{g^{2}}{2}a_{m}\wedge * a_{m} \right.\right. + a_{m}\wedge \left(\mathrm{d}a_{e} + a_{e}\wedge a_{e} \right) \nonumber \\
    &+ \left.\left.\frac{k}{4\pi}\left(a_{e}\wedge \mathrm{d}a_{e} + \frac{2}{3}a_{e}\wedge a_{e}\wedge a_{e} \right)\right)\right\}.
\end{align}
Now substituting in $a_{e} = b - \left(\frac{2\pi}{k}\right)a_{m}$, and noting that $\mathrm{d}a_{e} = \mathrm{d}b - \left(\frac{2\pi}{k}\right)\mathrm{d}a_{m}$, this leaves
\begin{align}
    \mathcal{Z}_{\text{Master}}= \int \mathcal{D}a_{m}\mathcal{D}b\exp i \text{tr}&\left\{\int \left(-\frac{g^{2}}{2}a_{m}\wedge * a_{m} \right.\right. \nonumber\\&+ a_{m}\wedge \left(\mathrm{d}\left(b - \left(\frac{2\pi}{k}\right)a_{m}\right) + \left(b - \left(\frac{2\pi}{k}\right)a_{m}\right)\wedge \left(b - \left(\frac{2\pi}{k}\right)a_{m}\right) \right) \nonumber \\
    &+ \frac{k}{4\pi}\left(\left(b - \left(\frac{2\pi}{k}\right)a_{m}\right)\wedge \mathrm{d}\left(b - \left(\frac{2\pi}{k}\right)a_{m}\right)\right. \nonumber+\\& \left.\left.\left.\frac{2}{3}\left(b - \left(\frac{2\pi}{k}\right)a_{m}\right)\wedge \left(b - \left(\frac{2\pi}{k}\right)a_{m}\right)\wedge \left( b - \left(\frac{2\pi}{k}\right)a_{m}\right) \right)\right)\right\}.
\end{align}
This expands to
\begin{align}
    \mathcal{Z}_{\text{Master}}=& \int \mathcal{D}a_{m}\mathcal{D}b\exp i \text{tr}\left\{\int \left(-\frac{g^{2}}{2}a_{m}\wedge * a_{m} \right.\right. \nonumber\\&+ a_{m}\wedge\mathrm{d}b - \left(\frac{2\pi}{k}\right)a_{m}\wedge \mathrm{d}a_{m} \nonumber \\
    &+ a_{m}\wedge b\wedge b -\left(\frac{2\pi}{k}\right)a_{m}\wedge a_{m} \wedge b - \left(\frac{2\pi}{k}\right)a_{m}\wedge b\wedge a_{m}+\left(\frac{2\pi}{k}\right)^{2}a_{m}\wedge a_{m}\wedge a_{m} \nonumber \\
    &+ \frac{k}{4\pi}b\wedge \mathrm{d}b-\frac{k}{4\pi}\left(\frac{2\pi}{k}\right)a_{m}\wedge \mathrm{d}b-\frac{k}{4\pi}\left(\frac{2\pi}{k}\right)b\wedge \mathrm{d}a_{m} + \frac{k}{4\pi}\left(\frac{2\pi}{k}\right)^{2}a_{m}\wedge \mathrm{d}a_{m}
    \nonumber\\&+\frac{k}{4\pi}\frac{2}{3}b\wedge b\wedge b -\frac{k}{4\pi}\frac{2}{3}\left(\frac{2\pi}{k}\right)b\wedge b \wedge a_{m} - \frac{k}{4\pi}\frac{2}{3}\left(\frac{2\pi}{k}\right)b\wedge a_{m}\wedge b + \frac{k}{4\pi}\frac{2}{3}\left(\frac{2\pi}{k}\right)^{2}b\wedge a_{m}\wedge a_{m} \nonumber \\
    &-\frac{k}{4\pi}\frac{2}{3}\left(\frac{2\pi}{k}\right)a_{m}\wedge b \wedge b + \frac{k}{4\pi}\frac{2}{3}\left(\frac{2\pi}{k}\right)^{2}a_{m}\wedge b\wedge a_{m} + \frac{k}{4\pi}\frac{2}{3}\left(\frac{2\pi}{k}\right)^{2}a_{m}\wedge a_{m}\wedge b -\nonumber\\
    &\left.\left.\frac{k}{4\pi}\frac{2}{3}\left(\frac{2\pi}{k}\right)^{3}a_{m}\wedge a_{m}\wedge a_{m}\right)\right\}.
\end{align}
Grouping terms in the same direction, but before permuting any wedge products, this gives
\begin{align}
    \mathcal{Z}_{\text{Master}}=& \int \mathcal{D}a_{m}\mathcal{D}b\exp i \text{tr}\left\{\int \left(-\frac{g^{2}}{2}a_{m}\wedge * a_{m} + a_{m}\wedge\mathrm{d}b\left(1-\frac{1}{2}\right)
    \right.\right. \nonumber\\
    &a_{m}\wedge \mathrm{d}a_{m}\left(- \left(\frac{2\pi}{k}\right)+ \frac{\pi}{k} \right)\nonumber \\
    &+ a_{m}\wedge b\wedge b-\frac{1}{3}b\wedge b \wedge a_{m}- \frac{1}{3}b\wedge a_{m}\wedge b-\frac{1}{3}a_{m}\wedge b \wedge b\nonumber\\
    &-\left(\frac{2\pi}{k}\right)a_{m}\wedge a_{m} \wedge b- \left(\frac{2\pi}{k}\right)a_{m}\wedge b\wedge a_{m}+ \frac{2\pi}{3k}b\wedge a_{m}\wedge a_{m}+ \frac{2\pi}{3k}a_{m}\wedge a_{m}\wedge b+ \frac{2\pi}{3k}a_{m}\wedge b\wedge a_{m} \nonumber \\
    &\left.\left.  +\left[\frac{4\pi^{2}}{k^{2}} -\frac{4\pi^{2}}{3k^{2}}\right]a_{m}\wedge a_{m}\wedge a_{m}+ \frac{k}{4\pi}b\wedge \mathrm{d}b-\frac{1}{2}b\wedge \mathrm{d}a_{m} 
    +\frac{k}{6\pi}b\wedge b\wedge b   \right)\right\}.
\end{align}
Using the result of equation \eqref{Eq: Trace result}, i.e., that
\begin{equation}
    \text{tr}\left(a\wedge b \wedge b\right) = \text{tr}\left(b\wedge a \wedge b\right) = \text{tr}
    \left(b\wedge b \wedge a\right),
\end{equation}
the above then simplifies to 
\begin{align}
    \mathcal{Z}_{\text{Master}}=& \int \mathcal{D}a_{m}\mathcal{D}b\exp i \text{tr}\left\{\int \left(-\frac{g^{2}}{2}a_{m}\wedge * a_{m} + \frac{1}{2}\left( a_{m}\wedge\mathrm{d}b-b\wedge \mathrm{d}a_{m}\right)-\frac{\pi}{k}a_{m}\wedge \mathrm{d}a_{m}
    \right.\right. \nonumber\\
    &+0 \times a_{m}\wedge b\wedge b-\frac{2\pi}{k}a_{m}\wedge b \wedge a_{m}\nonumber\\
    & \left.\left. +\frac{2}{3}\left(\frac{2\pi}{k}\right)^{2}a_{m}\wedge a_{m}\wedge a_{m}+ \frac{k}{4\pi}b\wedge \mathrm{d}b
    +\frac{k}{6\pi}b\wedge b\wedge b   \right)\right\}.
\end{align}
Now, a general result is that
\begin{equation}
    \mathrm{d}\left(\alpha\wedge \beta\right) = \left(\mathrm{d}\alpha\right)\wedge \beta + \left(-1\right)^{q}\wedge \left(\mathrm{d}\beta\right),\quad \alpha\in \Lambda^{q}\left(\mathcal{M}\right),\,\beta\in\Lambda^{r}\left(\mathcal{M}\right).
\end{equation}
So here, since $a_{m}$ and $b$ are both one-forms, the total derivative of their wedge product is
\begin{align}
    \mathrm{d}\left(a_{m}\wedge \mathrm{d}b\right) &= \left(\mathrm{d}a_{m}\right)\wedge b + \left(-1\right)^{1}a_{m}\wedge \mathrm{d}b\\
    &= \left(\mathrm{d}a_{m}\right)\wedge b -a_{m}\wedge \mathrm{d}b,
\end{align}
meaning the second term in the above expression for $\mathcal{Z}_{\text{Master}}$ is a total derivative and so can be omitted from the action. Dropping also the $a_{m}\wedge b \wedge b$ term gives
\begin{align}
    \mathcal{Z}_{\text{Master}}=& \int \mathcal{D}a_{m}\mathcal{D}b\exp i \text{tr}\left\{\int \left(-\frac{g^{2}}{2}a_{m}\wedge * a_{m} -\frac{\pi}{k}a_{m}\wedge \mathrm{d}a_{m}-\frac{2\pi}{k}a_{m}\wedge b \wedge a_{m}
    \right.\right. \nonumber\\
    & \left.\left. +\frac{2}{3}\left(\frac{2\pi}{k}\right)^{2}a_{m}\wedge a_{m}\wedge a_{m}+ \frac{k}{4\pi}b\wedge \mathrm{d}b
    +\frac{k}{6\pi}b\wedge b\wedge b   \right)\right\}.
\end{align}
The above is one way of expressing the new form of the partition function with all terms expanded. One way of combining these that looks somewhat more familiar is
\begin{align}
    \mathcal{Z}_{\text{Master}}= \int \mathcal{D}a_{m}\mathcal{D}b\exp i &\text{tr}\left\{\int \left(-\frac{g^{2}}{2}a_{m}\wedge * a_{m} -\frac{\pi}{k}a_{m}\wedge \left(\mathrm{d}a_{m}+2a_{m}\wedge \left(b-\frac{4\pi}{3k}a_{m}\right)
    \right)\right.\right. \nonumber\\
    & \left.\left. +\frac{k}{4\pi}\left(b\wedge \mathrm{d}b + \frac{2}{3}b\wedge b \wedge b\right) \right)\right\}.
\end{align}

\section{Notes on differential geometry}
For $p-$forms, write
\begin{equation}
    \omega = \frac{1}{p!}\omega_{\mu_{1}\ldots\mu_{p}}\omega_{\mu_{1}\ldots\mu_{p}}\mathrm{d}x^{\mu_{1}}\wedge \ldots\wedge\mathrm{d}x^{\mu_{p}},\quad \omega\in \Lambda^{p}\left(\mathcal{M}\right).
\end{equation}

\section{Notes on Lie-Algebra valued forms}
Writing a Lie-algebra valued one-form as 
\begin{equation}
    A = A_{\mu}\mathrm{d}x^{\mu},
\end{equation}
it is the case that $A_{\mu}$ can decomposed as 
\begin{equation}
    A_{\mu} = A_{\mu}^{a}T^{a},
\end{equation}
because the value of $A_{\mu}$ is contained within the Lie-algebra. Note also $A_{\mu}^{a} = A_{\mu}^{a}(x)$, and $x\in \mathbb{R}^{n}$. So really,
\begin{equation}
    A = A_{\mu}^{a}(x)\, T^{a}\otimes \mathrm{d}x^{\mu},
\end{equation}
with $T^{a}$ elements of the Lie algebra defined at each point that $x$ is evaluated for. \newline

\noindent Wedge products need to be taken care of. Can use the trick when traces are taken where the generators of the lie algebra get factorised away from the coefficients as 
\begin{align}
    \text{tr}\left(a\wedge b\wedge b \right)&= \text{tr}\left(a^{a}t^{a}\wedge b^{b}t^{b}\wedge b^{c}t^{c} \right)\nonumber \\
    &= \text{tr}\left(t^{a}t^{b}t^{c} \right)\left(a^{a}\wedge b^{b}\wedge b^{c} \right)\nonumber\quad \text{\color{blue}(Factorising trick) \color{black}}\\
    &= \text{tr}\left(t^{c}t^{a}t^{b} \right)a^{a}\wedge b^{b}\wedge b^{c} \nonumber\quad \text{\color{blue}(Trace cyclicity) \color{black}}\\
    &= -\text{tr}\left(t^{c}t^{a}t^{b} \right)a^{a}\wedge b^{c}\wedge b^{b} \nonumber\quad \text{\color{blue}(Wedge product antisymmetry) \color{black}}\\
    &= \text{tr}\left(t^{c}t^{a}t^{b} \right)b^{c}\wedge a^{a}\wedge b^{b} \nonumber\quad \text{\color{blue}(Wedge product antisymmetry) \color{black}}\\
    &=\text{tr}\left(b^{c}t^{c}\wedge a^{a}t^{a}\wedge b^{b}t^{b} \right)\nonumber \\
    &=  \text{tr}\left(b\wedge a\wedge b \right).
\end{align}
Similarly, find 
\begin{align}
    \text{tr}\left(a\wedge b\wedge b \right)&= \text{tr}\left(t^{c}t^{a}t^{b} \right)b^{c}\wedge a^{a}\wedge b^{b} \nonumber\nonumber\quad \text{\color{blue}(From above) \color{black}}\\
    &= \text{tr}\left(t^{b}t^{c}t^{a}\right)b^{c}\wedge a^{a}\wedge b^{b} \nonumber\quad \text{\color{blue}(Trace cyclicity) \color{black}}\\
    &= - \text{tr}\left(t^{b}t^{c}t^{a}\right)b^{c}\wedge b^{b}\wedge a^{a} \nonumber\quad \text{\color{blue}(Wedge product antisymmetry) \color{black}}\\
    &= \text{tr}\left(t^{b}t^{c}t^{a}\right) b^{b}\wedge b^{c}\wedge a^{a} \nonumber\quad \text{\color{blue}(Wedge product antisymmetry) \color{black}}\\
    &= \text{tr}\left(b^{b}t^{b}\wedge b^{c}t^{c}\wedge a^{a}t^{a}\right)\\
    &= \text{tr}\left(b\wedge b \wedge a \right).
\end{align}
And so
\begin{equation}
    \label{Eq: Trace result}
    \boxed{
    \text{tr}\left(a\wedge b \wedge b\right) = \text{tr}\left(b\wedge a \wedge b\right) = \text{tr}
    \left(b\wedge b \wedge a\right).
    }
\end{equation}
\section{Dimensional reduction}
To reduce from 4D coordinates in $\mathbb{R}^{4}$ coordinates to 3D, first constrain the fields $\mathbb{A}_{\mu}(x)$ to be defined on $\mathbb{R}^{3}\times I$, where $I$ is some interval. Now things parameterised by $x\in \mathbb{R}^{3}$ and $\sigma\in I$. Eventually see that a 4D vector reduces to a 3D vector plus one scalar. \newline 

\noindent Take lowest mode as only it satisfies the boundary condition. This mode is either the vector or the scalar in our case. In 4D, $A_{\mu} \rightarrow A'_{\mu'} = \Lambda^{\mu}_{\,\mu '}A_{\mu}$. In 3D, we need a vector in a 3D subspace to remain in that 3D subspace under the transformations we are allowing.\newline


\noindent To move from $D$ dimensions to $d$ dimensions, we make the $D-d$ dimensions we wish to remove compact, and then require that in the limit that their size goes to zero, that the energy remains finite. \newline

\noindent For example, for a scalar field, consider $\phi\left(x\right)$ with a compact dimensions with period $L$. Can write any field of this sort as
\begin{equation}
    \phi_{n}\left(x\right) = A_{n}\cos\left(\frac{2\pi n x}{L}\right).
\end{equation}
According to quantum mechanics, this has a momentum $\pm nh/L$ along the $x$ direction. Therefore, unless only the $n=0$ mode is non-zero as $L\rightarrow \infty$, then the momentum in this direction diverges. However, $\partial_{x}\phi_{0}=0$, and also $\partial_{x}\phi_{x}\neq 0$ for $n\neq 0$, and so removal of the $\phi$ dependence on $x$ gives the desired behaviour. We still require the fields we do things like this to to transform correctly under new subgroups of the symmetry groups we had before, however this will be discussed later. \newline

\subsection{Dimensional reduction of electromagnetism}
Following the work of the paper \href{https://arxiv.org/pdf/2111.10728.pdf}{Dimensional reduction of electromagnetism}, start with the Lagrangian of (3+1)-dimensional electromagnetism
\begin{equation}
    \mathcal{L}_{3+1}\left(A^{\mu}\right) = -\frac{1}{4}F^{\mu\nu}F_{\mu\nu} - \frac{1}{c}j^{\mu}A_{\mu}.
\end{equation}
Here, the components of the field strength in Cartesian coordinates are
\begin{equation}
    \left(F_{\mu\nu}\right) = \begin{pmatrix}
        0 & -E_x & -E_y & -E_z \\
        E_x & 0 & -B_z & B_y \\
        E_y & B_z & 0 & -B_x \\
        E_z & -B_y & B_x & 0 \\
    \end{pmatrix}, 
\end{equation}
and the corresponding components of the current are
\begin{equation}
    J^\mu = \begin{pmatrix}
        c \rho \\
        J_{x} \\
        J_{y} \\
        J_{z}
    \end{pmatrix}.
\end{equation}
This gives the Lagrangian in terms of the electric and magnetic fields as
\begin{equation}
    \mathcal{L} = \frac{1}{2}\left(\vec{E}\cdot\vec{E} - \vec{B}\cdot\vec{B}\right)-\rho\phi + \frac{1}{c}\vec{J}\cdot\vec{A}.
\end{equation}
Note here can also define the dual field strength tensor
\begin{equation}
    *F^{\mu\nu} \equiv G^{\mu\nu} := \frac{1}{2}\varepsilon^{\mu\nu\rho\sigma}F_{\rho\sigma},
\end{equation}
which in turn gives back the field strength tensor via
\begin{equation}
    F^{\mu\nu} = -\frac{1}{2}\varepsilon^{\mu\nu\rho\sigma}G_{\rho\sigma}.
\end{equation}
This gives the components
\begin{equation}
    *\!F_{\mu\nu} = \begin{pmatrix}
        0 & -B_x & -B_y & -B_z \\
        B_x & 0 & E_z & -E_y \\
        B_y & -E_z & 0 & E_x \\
        B_z & E_y & -E_x & 0 \\
    \end{pmatrix}.
\end{equation}
Note that these can be organised into blocks;
\begin{equation}
    F_{\mu\nu} = \begin{pmatrix}
        \begin{array}{c|ccc}
            0 & -E_x & -E_y & -E_z \\ \hline
            E_x & 0 & -B_z & B_y \\
            E_y & B_z & 0 & -B_x \\
            E_z & -B_y & B_x & 0 \\
        \end{array}
    \end{pmatrix} = \begin{pmatrix}
        \begin{array}{c|c}
            & -\vec{E}^T \\ \hline
            \vec{E} & \tilde{B}
        \end{array}
    \end{pmatrix},
\end{equation}
\begin{equation}
    G_{\mu\nu} = \begin{pmatrix}
        0 & -B_x & -B_y & -B_z \\
        B_x & 0 & E_z & -E_y \\
        B_y & -E_z & 0 & E_x \\
        B_z & E_y & -E_x & 0 \\
    \end{pmatrix} = \begin{pmatrix}
        \begin{array}{c|c}
            & -\vec{B}^T \\ \hline
            \vec{B} & \tilde{E}
        \end{array}
    \end{pmatrix}
\end{equation}
and
\begin{equation}
    J^\mu = \begin{pmatrix}
        c \rho \\
        J_{x} \\
        J_{y} \\
        J_{z}
    \end{pmatrix}= \begin{pmatrix}
        c \rho \\
        \hline
        \\
        \vec{J}\\
        \\
    \end{pmatrix}.
\end{equation}
We are interested in proper Lorentz transformations $SO^{+}\left(1,\,3\right)$. This is the set of all Lorentz transformations that can be connected to the identity by a continuous curve lying in the group. This is composed of proper rotations and boosts. 
The set of all rotations forms a Lie subgroup isomorphic to the ordinary rotation group SO(3). These rotations leave $\vec{E}$ and $\vec{B}$ unmixed, and so their $3\oplus 3$ structure preserved. Similarly, $\rho$ and $\vec{J}$ are left unmixed by these transformations and so their $1\oplus 3$ block structure is preserved too. \newline

\noindent The equations constraining electromagnetism can be derived from the Lagrangian and are as following: 
\begin{align}
    &\text{Dynamics: }\partial_{\mu}F^{\mu\nu} = \frac{1}{c}j^{\nu},\quad \partial_{\mu}G^{\mu\nu}=0,\\
    &\text{Continuity: }\partial_{t}\rho + \partial_{x}J_{x} + \partial_{y}J_{y} + \partial_{z}J_{z}=0.
\end{align}
Breaking the field strength tensor and its dual into $\vec{E}$ and $\vec{B}$ gives the Maxwell equations in their familiar form
\begin{align}
    \text{div}\vec{B}&=0 \quad \text{(Gauss's law for magnetism),}\label{Gauss law magnetism}\\
    \frac{1}{c}\partial_{t}\vec{B} + \text{curl}\vec{E} &= \vec{0}\quad \text{(Maxwell-Faraday law of induction),}\label{Maxwell Faraday law}\\
    \text{div}\vec{E} &= \rho \quad \text{(Gauss's law),}\label{Gauss law}\\
    -\frac{1}{c}\partial_{t}\vec{E} + \text{curl}\vec{B} &= \frac{1}{c}\vec{J}\quad\text{(Ampère's circuital law).\label{Ampere's law}}
\end{align}
Breaking these equations into Cartesian components gives the following set of equations. Firstly, \eqref{Gauss law magnetism} implies
\begin{equation}
    \partial_{x}B_{x}+\partial_{y}B_{y}+\partial_{z}B_{z}=0.
\end{equation}
Equation \eqref{Maxwell Faraday law} implies the three equations
\begin{align}
    \frac{1}{c}\partial_{t}B_{x}+\partial_{y}E_{z}-\partial_{z}E_{y}&=0,\\
    \frac{1}{c}\partial_{t}B_{y}+\partial_{z}E_{x}-\partial_{x}E_{z}&=0,\\
    \frac{1}{c}\partial_{t}B_{z}+\partial_{x}E_{y}-\partial_{y}E_{x}&=0.
\end{align}
Next, equation \eqref{Gauss law} implies
\begin{equation}
    \partial_{x}E_{x}+\partial_{y}E_{y}+\partial_{z}E_{z}=\rho.
\end{equation}
Finally, equation \eqref{Ampere's law} implies the three equations
\begin{align}
    -\frac{1}{c}\partial_{t}E_{x}+\partial_{y}B_{z}-\partial_{z}B_{y}&=\frac{1}{c}J_{x},\\
    -\frac{1}{c}\partial_{t}E_{y}+\partial_{z}B_{x}-\partial_{x}B_{z}&=\frac{1}{c}J_{y},\\
    -\frac{1}{c}\partial_{t}E_{z}+\partial_{x}B_{y}-\partial_{y}B_{x}&=\frac{1}{c}J_{z}.
\end{align}
Now, a \textit{descent} is performed along the $z$ direction. This is done by requiring the z-independence of both sources and solutions. This is achieved by letting $\partial_{z}\rightarrow 0$. The result is that the equations of motion decouple into two independent sets of four equations. These sets are:\newline
\subsubsection*{1: The $\mathbf{\left(E_{x},\,E_{y},\,B_{z}\right),\,\left(\rho,\,J_{x},\,J_{y}\right)}$ sector}
This sector is made of 4 differential equations and a reduced continuity equation:
\begin{align}
    \frac{1}{c}\partial_{t}B_{z}+\partial_{x}E_{y}-\partial_{y}E_{x}&=0,\\
    \partial_{x}E_{x}+\partial_{y}E_{y}&=\rho,\\
    -\frac{1}{c}\partial_{t}E_{x}+\partial_{y}B_{z}&=\frac{1}{c}J_{x},\\
    -\frac{1}{c}\partial_{t}E_{y}-\partial_{x}B_{z}&=\frac{1}{c}J_{y},\\
\end{align}
and \begin{equation}
    \partial_{t}\rho+\partial_{x}J_{x}+\partial_{y}J_{y}=0.
\end{equation}

\subsubsection*{2: The $\mathbf{\left(B_{x},\,B_{y},\,E_{z}\right),\,\left(J_{z}\right)}$ sector}
This section is made of only 4 differential equations
\begin{align}
    \partial_{x}B_{x}+\partial_{y}B_{y}+\partial_{z}B_{z}&=0,\\
    \frac{1}{c}\partial_{t}B_{x}+\partial_{y}E_{z}&=0,\\
     \frac{1}{c}\partial_{t}B_{y}-\partial_{x}E_{z}&=0,\\
     -\frac{1}{c}\partial_{t}E_{z}+\partial_{x}B_{y}-\partial_{y}B_{x}&=\frac{1}{c}J_{z}.
\end{align}
This partitioning can be highlighted in the field strength tensors
\begin{equation}
    F_{\mu\nu} = \begin{pmatrix}
        \begin{array}{ccc|c}
            0 & \color{blue}-E_x \color{black}& \color{blue}-E_y \color{black}& \color{red}-E_z\color{black} \\
            \color{blue}E_x \color{black}& 0 & \color{blue}-B_z\color{black} & \color{red}B_y \color{black}\\
            \color{blue}E_y \color{black}& \color{blue}B_z\color{black} & 0 & \color{red}-B_x\color{black} \\\hline
            \color{red}E_z\color{black} & \color{red}-B_y\color{black} & \color{red}B_x\color{black} & 0 \\
        \end{array}
    \end{pmatrix},\quad
    G_{\mu\nu} = \begin{pmatrix}
        \begin{array}{ccc|c}
            0 & \color{red}-B_x \color{black}& \color{red}-B_y \color{black}& \color{blue}-B_z\color{black} \\
            \color{red}B_x \color{black}& 0 & \color{red}E_z\color{black} & \color{blue}-E_y \color{black}\\
            \color{red}B_y \color{black}& \color{red}-E_z\color{black} & 0 & \color{blue}E_x\color{black} \\\hline
            \color{blue}B_z\color{black} & \color{blue}E_y\color{black} & \color{blue}-E_x\color{black} & 0 \\
        \end{array}
    \end{pmatrix}, \quad
    J^\mu = \begin{pmatrix}
        \color{blue}c \rho \color{black}\\
        \color{blue}J_{x}\color{black}\\
        \color{blue}J_{y}\color{black}\\\hline
        \color{red}J_{z}\color{black}\\
    \end{pmatrix}.
\end{equation}
Note now that the elements of the first sector in $G$ can be obtained by performing the correct alternating sum over the elements of the first sector in F. This done in terms of reduced spacetime indices $a,\,b\in \left\{0,\,1,\,2\right\}$ as
\begin{equation}
    \color{red}G^{ab} = \frac{1}{2}\left(\varepsilon^{abc3}F_{c3} + \varepsilon^{ab3c}F_{3c}\right)\color{black},
\end{equation}
which reduces to 
\begin{equation}
    G^{ab} = \varepsilon^{abc}F_{c3}.
\end{equation}
Similarly, the second sector of $G$ can be obtained from the second sector in $F$ via
\begin{equation}
    \color{blue}G^{a3} = \frac{1}{2}\varepsilon^{a3bc}F_{bc}\color{black},
\end{equation}
which reduces to
\begin{equation}
    G^{a3} = \frac{1}{2}\varepsilon^{abc}F_{bc}.
\end{equation}
By recognising that the field strength tensors partition this way, the equations of motion for each sector can be written in tensor form. For the first (EEB) sector, these are
\begin{equation}
    \boxed{
    \varepsilon^{abc}\partial_{a}F_{bc}=0,\,\quad \partial_{a}F^{ab} = \frac{1}{c}j^{b}
    }
\end{equation}
Now in this sector, $F^{ab}$ is an antsymmetric (2+1) dimensional tensor, and its dual, $G^{a3}$ is a (2+1) dimensional vector. Also, $j^{a}$ is a (2+1) dimensional vector. This sector is considered the closer analog of 4D electromagnetism in 3D. \newline

\noindent In the second (BBE) sector, the equations of motion in tensor form are
\begin{equation}
    \boxed{
    \varepsilon^{abc}\partial_{b}F_{c3}=0,\,\quad \partial_{a}F^{a3}= \frac{1}{c}j^{3}.
    }
\end{equation}
In this sector, $F^{a3}$ is a 3-component vector, and its dual $G^{ab}$ is a (2+1) dimensional antisymmetric tensor. Finally note that $j^{3}$ is a scalar.\newline

\noindent Now that a partitioning of the field components into sectors has been made clear, it is important to check that these are not mixed due to the application of the desired (Lorentz) transformations. Since $z$-dependence has been removed from the physics, the subgroup of Lorentz transformations that leaves the z components of vectors preserved is now the set of transformations of interest. This subgroup is represented in Cartesian coordinates by the set of matrices with the block diagonal form
\begin{equation}
    \Lambda =  \begin{pmatrix}
        \begin{array}{c|c}
            L&\\ \hline
             & Q
        \end{array}
    \end{pmatrix},
\end{equation}
where $L\in O\left(2,\,1\right)$ and $Q\in O\left(1\right)=\left\{+1,\,-1\right\}$. When $Q=+1$, this leaves the $z$-axis preserved, whereas $Q=-1$ inverts the $z$-axis. It can be checked that applications of these matrices leave the sectors unmixed. \newline

\noindent The above means that under dimensional reduction, $F^{\mu\nu}$ has become the (2+1)-dimensional two-form $F^{ab}$ and the one form $F^{a3}$. Likewise, $A^{\mu}$ has become the 3D gauge field $A^{a}$ and the scalar $A^{3}$. This is a key initial result used in Adi's paper. \newline

\noindent Additionally, the original Lagrangian $\mathcal{L}_{3+1}\left(A\right)$ can be written
\begin{equation}
    \mathcal{L}_{3+1}\left(A\right) = \left(-\frac{1}{4}F_{ab}F^{ab} - \frac{1}{c}j^{a}A_{a} \right) + \left(-\frac{1}{2}F_{a3}F^{a3} - \frac{1}{c}j^{3}A_{3}\right).
\end{equation}
Despite how things might look, this is not fully decoupled into $A^{a}$ and $A^{3}$ in the way the derivatives of $A$ are. First it is required that the gauge choice
\begin{equation}
    \partial_{3}A_{a}=0
\end{equation}
(which can always be made) be made. This choice leaves the residual gauge freedom
\begin{equation}
    A_{a}\rightarrow A_{a}+\partial_{a}f,\quad f=f\left(t,\,x,\,y\right). 
\end{equation}
Under this choice, the full Lagrangian can be written
\begin{equation}
    \mathcal{L}_{3+1}\left(A\right) = \mathcal{L}_{\text{EEB}}\left(A^{0},\,A^{1},\,A^{2}\right) + \mathcal{L}_{\text{BBE}}\left(A^{3}\right),
\end{equation}
where
\begin{align}
    \mathcal{L}_{\text{EEB}}\left(A^{0},\,A^{1},\,A^{2} \right) &= -\frac{1}{4}\left(\partial_{a}A_{b}-\partial_{b}A_{a}\right)\left(\partial^{a}A^{b} - \partial^{b}A^{a}\right) - \frac{1}{c}j^{a}A_{a}\\
    &= \frac{1}{2}\left(E_{x}^{2}+E_{y}^{2}-B_{z}^{2}\right) - \rho\phi + \frac{1}{c}\left(J_{x}A_{x}+J_{y}A_{y}\right),
\end{align}
and 
\begin{align}
    \mathcal{L}_{BBE}\left(A^{3}\right) &= -\frac{1}{2}\left(\partial_{a}A_{3}\right)\left(\partial^{a}A^{3}\right) - \frac{1}{c}j^{3}A_{3}\\
    &= \frac{1}{2}\left(E_{z}^{2} - B_{x}^{2}-B_{y}^{2}\right) + \frac{1}{c}J_{z}A_{z},
\end{align}
which is truly decoupled into two pieces, with a full gauge freedom in the non-compact directions remaining.



\section{Behaviour of $Z_{M}$ under gauge transformations}
\label{Sec: Behaviour of Z_{M} under gauge transformations}
The term of most interest in the partition function \eqref{Eq: Expanded magnetic} is $\text{tr}\left(a_{m}\wedge a_{m}\wedge b\right)$. Before taking this on, note
\begin{align}
     0&= \mathrm{d}\left(\mathbf{1}\right)\\
     &= \mathrm{d}\left(g\,g^{\ddagger}\right)\\
    &= \mathrm{d}\left(g\right)g^{\dagger} + g\mathrm{d}\left(g^{\dagger}\right).
\end{align}
Knowing this, and performing the gauge transformation $a_{m}\rightarrow ga_{m}g^{\dagger}$, $b\rightarrow gbg^{\dagger} + g\mathrm{d}g^{\dagger}$, the term of interest transforms as
\begin{align}
    \text{tr}\left(a_{m}\wedge a_{m}\wedge b\right) &\rightarrow \text{tr}\left(g a_{m} g^{\dagger}\wedge ga_{m}g^{\dagger}\wedge\left(gbg^{\dagger}+g\mathrm{d}g^{\dagger}\right)\right)\\
    &=\text{tr}\left(ga_{m}g^{\dagger}\wedge ga_{m}g^{\dagger}\wedge gbg^{\dagger}\right) + \text{tr}\left(ga_{m}g^{\dagger}\wedge ga_{m}g^{\dagger}\wedge g\mathrm{d}g^{\dagger}\right)\\
    &= \text{tr}\left(a_{m}\wedge a_{m}\wedge b\right) +\text{tr}\left(ga_{m}\wedge a_{m}\wedge g\mathrm{d}g^{\dagger}\right)\\
    &= \text{tr}\left(a_{m}\wedge a_{m}\wedge b\right)-\text{tr}\left(ga_{m}\wedge a_{m}\wedge \left(\mathrm{d}g\right)g^{\dagger}\right)\\
    &= \text{tr}\left(a_{m}\wedge a_{m}\wedge b\right)-\text{tr}\left(a_{m}\wedge a_{m}\wedge \mathrm{d}g\right)\\
\end{align}
So the variation in this term is
\begin{equation}
    \Delta\left(a_{m}\wedge a_{m}\wedge b\right) = \text{tr}\left(a_{m}\wedge a_{m}\wedge \mathrm{d}g \right)
\end{equation}
Using then that $a_{m}$ is a 1-form, consider the total derivative $\mathrm{d}\left(a_{m}\wedge a_{m}\wedge g\right)$. This expands as
\begin{align}
    \mathrm{d}\left(a_{m}\wedge \left(a_{m}\wedge g\right)\right)&= \mathrm{d}a_{m}\wedge \left(a_{m}\wedge g\right) - a_{m}\wedge\mathrm{d}\left(a_{m}\wedge g\right)\\
    &=\mathrm{d}a_{m}\wedge \left(a_{m}\wedge g\right) - a_{m}\wedge \left(\mathrm{d}a_{m}\wedge g- a_{m}\wedge \mathrm{d}g\right)\\
    &= \mathrm{d}a_{m}\wedge \left(a_{m}\wedge g\right) - a_{m}\wedge \left(\mathrm{d}a_{m}\wedge g\right) + a_{m}\wedge a_{m}\wedge\mathrm{d}g\\
    &= \left(\mathrm{d}a_{m}\wedge a_{m}-a_{m}\wedge \mathrm{d}a_{m}\right)\wedge g + a_{m}\wedge a_{m}\wedge\mathrm{d}g\\
    &= \mathrm{d}\left(a_{m}\wedge a_{m}\right)\wedge g+ a_{m}\wedge a_{m}\wedge\mathrm{d}g.
\end{align}
And so the variation of $a_{m}\wedge a_{m}\wedge b$ under the gauge transformation is
\begin{align}
    \Delta\left(a_{m}\wedge a_{m}\wedge b\right) &= \text{tr}\left(a_{m}\wedge a_{m}\wedge \mathrm{d}g \right)\\
    &= \text{tr}\left(\mathrm{d}\left(a_{m}\wedge \left(a_{m}\wedge g\right)\right)\right) - \text{tr}\left(\mathrm{d}\left(a_{m}\wedge a_{m}\right)\wedge g\right).
\end{align}
Since total derivatives don't contribute to the action, this variation is then equivalent to
\begin{equation}
    \Delta\left(a_{m}\wedge a_{m}\wedge b\right) \equiv - \text{tr}\left(\mathrm{d}\left(a_{m}\wedge a_{m}\right)\wedge g\right)
\end{equation}
Since $g$ is a 0-form, this is then just
\begin{equation}
    \Delta\left(a_{m}\wedge a_{m}\wedge b\right) = - \text{tr}\left(\mathrm{d}\left(a_{m}\wedge a_{m}\right)\cdot g\right),
\end{equation}
where $\cdot$ denotes the (matrix) product of $\mathrm{d}\left(a_{m}\wedge a_{m}\right)$ and $g$. 
\end{document}
